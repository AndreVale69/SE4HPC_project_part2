\documentclass[a4paper]{article}
\usepackage[T1]{fontenc}			% \chapter package
\usepackage[english]{babel}
\usepackage[english]{isodate}  		% date format
\usepackage{graphicx}				% manage images
\usepackage{amsfonts}
\usepackage{booktabs}				% high quality tables
\usepackage{amsmath}				% math package
\usepackage{amssymb}				% another math package (e.g. \nexists)
\usepackage{bm}                     % bold math symbols
\usepackage{mathtools}				% emphasize equations
\usepackage{stmaryrd} 				% '\llbracket' and '\rrbracket'
\usepackage{amsthm}					% better theorems
\usepackage{enumitem}				% manage list
\usepackage{pifont}					% nice itemize
\usepackage{cancel}					% cancel math equations
\usepackage{caption}				% custom caption
\usepackage[]{mdframed}				% box text
\usepackage{multirow}				% more lines in a table
\usepackage{textcomp, gensymb}		% degree symbol
\usepackage[x11names]{xcolor}		% RGB color
\usepackage[many]{tcolorbox}		% colorful box
\usepackage{multicol}				% more rows in a table (used for the lists)
\usepackage{listings}
\usepackage{url}
\usepackage{qrcode}
\usepackage{fontawesome5}
\usepackage{ragged2e}
\usepackage{cite}                   % references
\usepackage{imakeidx}               % index


\definecolor{codegreen}{rgb}{0,0.6,0}
\definecolor{codegray}{rgb}{0.5,0.5,0.5}
\definecolor{codepurple}{rgb}{0.58,0,0.82}
\definecolor{backcolour}{rgb}{0.95,0.95,0.92}
\lstdefinestyle{mystyle}{
    backgroundcolor=\color{backcolour},   
    commentstyle=\color{codegreen},
    keywordstyle=\color{magenta},
    numberstyle=\tiny\color{codegray},
    stringstyle=\color{codepurple},
    basicstyle=\ttfamily\footnotesize,
    breakatwhitespace=false,         
    breaklines=true,                 
    captionpos=b,                    
    keepspaces=true,                 
    numbers=left,                    
    numbersep=5pt,                  
    showspaces=false,                
    showstringspaces=false,
    showtabs=false,                  
    tabsize=2
}
\lstset{style=mystyle}


% table of content links
\usepackage{xcolor}
\usepackage[linkcolor=black, citecolor=blue, urlcolor=cyan]{hyperref} % hypertexnames=false
\hypersetup{
	colorlinks=true
}


\newcommand{\dquotes}[1]{``#1''}
\newcommand{\highspace}{\vspace{1.2em}\noindent}


\begin{document}
    \noindent
    Information:
    \begin{itemize}
        \item Andrea Valentini
        \item ID: 260236
        \item [\faIcon{github}] \href{https://github.com/AndreVale69/SE4HPC_project_part1}{SE4HPC\_project\_part1}
        \item [\faIcon{github}] \href{https://github.com/AndreVale69/SE4HPC_project_part2}{SE4HPC\_project\_part2}
        \item If the links are broken, this is my link to my GitHub profile (\href{https://github.com/AndreVale69}{GitHub AndreVale69}), the projects are in the repository list.
    \end{itemize}
    I was working alone because I'm a working student. But on Monday the 3rd I came to the lecture and with the great help of Simone Reale we solved a problem with the mpi on the Galileo100 cluster. I don't know if I was the first to finish the project or not, but as soon as I finished the project, I helped my colleagues.

    \highspace
    There are a few points in the following list that I would like to highlight:
    \begin{itemize}
        \item Part 1:
        \begin{itemize}
            \item I added some test cases to test the matrix multiplication in general and not with the aim to find the 20 errors. By the way, I found all 20 errors (check the comments in any function to find the errors it generates).
            
            \item I've created a CI/CD pipeline (optional requirement), but obviously the test doesn't pass and the result is an error.
        \end{itemize}
        
        \item Part 2:
        \begin{itemize}
            \item The \texttt{singularity-alpine.def} file is an attempt to create a container using \href{https://alpinelinux.org/about/}{Alpine Linux}. The main reason is lightweight. Usually in production, if it's possible, it's a good choice to make a container very lightweight. The main problem is that open mpi is still under development, so it is not possible to run it on the cluster.
            
            \item The \texttt{singularity.def} file is the main singularity container and is used for the cluster. The chosen operating system is Debian slim, because it's weight is only 75 MB. As the \href{https://hub.docker.com/_/debian}{documentation} says:
            \begin{center}
                \emph{These tags are an experiment in providing a slimmer base (removing some extra files that are normally not necessary within containers, such as man pages and documentation), and are definitely subject to change.}

                \emph{See the \texttt{debuerreotype-slimify} script (\texttt{debuerreotype} linked above) for more details about what gets removed during the \dquotes{slimification} process.}
            \end{center}

            \item The pipeline is divided into four main groups to take advantage of some interesting features of GitHub Actions, such as \textbf{GitHub Actions Cache}, \textbf{GitHub Marketplace actions}, \textbf{download/upload artifacts}:
            \begin{enumerate}
                \item \texttt{build-and-test}: Install the \texttt{libopenmpi-dev} library, import the submodule (\texttt{GoogleTest}), build the project and run the tests.

                \item \texttt{singularity}: Import the submodule (\texttt{GoogleTest}), check if Go is cached, otherwise download it using the action from the marketplace \texttt{actions/setup-go@v5.0.1}. Then download the Singularity dependencies, download the release from the official repo and install it using \texttt{dpkg}. Finally, create the container from the def file and upload it to \texttt{actions/upload-artifact@v4.3.3}.
                
                \item \texttt{upload-container-to-cluster}: Download the SIF container from \texttt{actions/download-artifact@v4.1.7}. Then create a zip file containing the minimum required to run on the Galileo100 cluster. Finally, use \texttt{sshpass} and \texttt{rsync} to pass \texttt{job.sh} and the zip file to the cluster.
                
                \item \texttt{run-container-to-cluster}: Run the unzip and delete the zip when it is finished. Finally, run \texttt{sbatch job.sh} to run it on the cluster. At the end, download \texttt{output.log} and \texttt{errors.log} and upload them as artifacts.
            \end{enumerate}
            Note: they run sequentially (pipeline); if one of them breaks, the pipeline breaks and github actions return an error.
        \end{itemize}
    \end{itemize}
\end{document}